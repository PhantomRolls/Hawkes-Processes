\documentclass[10pt]{beamer}

% --------------------
% Theme
% --------------------
\usetheme{Madrid}
\usecolortheme{default}
\setbeamertemplate{navigation symbols}{}

% --------------------
% Packages
% --------------------
\usepackage{amsmath, amssymb}
\usepackage{graphicx}
\usepackage{booktabs}

% --------------------
% Title info
% --------------------
\title[GOF for Hawkes]{Goodness-of-Fit Testing for Hawkes Processes}
\subtitle{Reproduction and Application of an Asymptotically Distribution-Free Test}
\author{Adrien Damez \and Arnaud Van de Velde \and Elena Perotti \\ Angelo Picerno \and Gavriil Kharkhodine}
\institute{Project P15 — Hawkes Processes}
\date{January 2026}

% ============================================================
\begin{document}

% ------------------------------------------------------------
\begin{frame}
  \titlepage
\end{frame}

% ------------------------------------------------------------
\begin{frame}{Motivation}
\begin{itemize}
  \item Hawkes processes are standard for \textbf{self-exciting event data}
  \item Widely used in:
  \begin{itemize}
    \item High-frequency finance
    \item Seismology, epidemiology, social networks
  \end{itemize}
  \item \alert{Key question}: how to assess whether a fitted Hawkes model is adequate?
\end{itemize}
\end{frame}

% ------------------------------------------------------------
\begin{frame}{Classical Goodness-of-Fit}
\begin{itemize}
  \item Standard approach: \textbf{Random Time Change (RTC)}
  \[
  \tau_i = \int_0^{t_i} \lambda_{\hat\theta}(s)\,ds
  \]
  \item Under the true model: transformed inter-arrival times $\sim \text{Exp}(1)$
  \item In practice:
  \begin{itemize}
    \item Parameters unknown
    \item Plug-in estimation error ignored
  \end{itemize}
  \item \alert{Consequence}: tests may be badly sized or overly sensitive
\end{itemize}
\end{frame}

% ------------------------------------------------------------
\begin{frame}{Main Idea of the Project}
\begin{block}{Reference}
Baars, Can \& Laeven (2025)
\end{block}

\begin{itemize}
  \item Propose an \textbf{asymptotically distribution-free} GOF test
  \item Explicitly accounts for:
  \begin{itemize}
    \item Parameter estimation uncertainty
  \end{itemize}
  \item Based on a \textbf{martingale transformation}
\end{itemize}

\vspace{0.2cm}
\textbf{Our goals}:
\begin{itemize}
  \item Reproduce their results for Hawkes processes
  \item Compare with naive and RTC-based procedures
  \item Apply to real high-frequency financial data
\end{itemize}
\end{frame}

% ------------------------------------------------------------
\begin{frame}{Hawkes Process Model}
\[
\lambda_\theta(t) = \mu + \sum_{t_i < t} g_\theta(t - t_i)
\]

\begin{itemize}
  \item $\mu > 0$: baseline intensity (constant)
  \item $g_\theta$: excitation kernel
  \item Stationarity condition:
  \[
  \int_0^\infty g_\theta(s)\,ds < 1
  \]
\end{itemize}

\textbf{Kernels considered}:
\begin{itemize}
  \item Exponential
  \item Power-law
  \item Multi-exponential
\end{itemize}
\end{frame}

% ------------------------------------------------------------
\begin{frame}{Parameter Estimation}
\begin{itemize}
  \item Parameters estimated by \textbf{Maximum Likelihood}
  \[
  \ell_T(\theta) = \sum_{i=1}^{N(T)} \log \lambda_\theta(t_i)
  - \int_0^T \lambda_\theta(t)\,dt
  \]
  \item Efficient recursion for exponential / multi-exponential kernels
  \item Power-law kernel: higher computational cost
\end{itemize}

\vspace{0.2cm}
\alert{Important}: treating $\hat\theta$ as fixed can distort GOF tests
\end{frame}

% ------------------------------------------------------------
\begin{frame}{Goodness-of-Fit Procedures Compared}
\begin{enumerate}
  \item \textbf{Naive compensated process}
  \item \textbf{Transformation-based procedure} (Baars et al.)
  \item \textbf{Naive Random Time Change (RTC)}
\end{enumerate}

\vspace{0.2cm}
All tests use classical statistics:
\begin{itemize}
  \item Kolmogorov–Smirnov
  \item Cramér–von Mises
  \item Anderson–Darling
\end{itemize}
\end{frame}

% ------------------------------------------------------------
\begin{frame}{Transformation-Based Test (Key Idea)}
\begin{itemize}
  \item Start from the compensated empirical process
  \[
  \eta_T(u) = \frac{1}{\sqrt{T}}
  \left(
  N(uT) - \int_0^{uT} \lambda_{\hat\theta}(s)\,ds
  \right)
  \]
  \item Apply a transformation that removes estimation effects
  \item Obtain increments that are asymptotically i.i.d. $\mathcal N(0,1)$
\end{itemize}

\begin{block}{Key advantage}
Asymptotic distribution does \textbf{not depend} on $\theta$
\end{block}
\end{frame}

% ------------------------------------------------------------
\begin{frame}{Simulation Study}
\begin{itemize}
  \item Hawkes processes simulated under the null
  \item Same setup as Baars et al. (2025)
  \item $T = 5000$, 500 Monte Carlo replications
  \item Compare rejection frequencies at 1\%, 5\%, 20\%
\end{itemize}

\vspace{0.2cm}
Models tested:
\begin{itemize}
  \item Correct specification
  \item Misspecified kernels
\end{itemize}
\end{frame}

% ------------------------------------------------------------
\begin{frame}{Simulation Results — Main Message}
\begin{itemize}
  \item \textbf{Transformation-based test}
  \begin{itemize}
    \item Correctly sized under the null
    \item Robust across kernel families
  \end{itemize}
  \item \textbf{Naive test}
  \begin{itemize}
    \item Strongly conservative
  \end{itemize}
  \item \textbf{Naive RTC}
  \begin{itemize}
    \item Undersized under the null
    \item Rejects aggressively under misspecification
  \end{itemize}
\end{itemize}
\end{frame}

% ------------------------------------------------------------
\begin{frame}{Application to Real Data}
\begin{itemize}
  \item High-frequency trade data (Société Générale)
  \item 12 trading days, 9:00–17:30
  \item Data split into \textbf{one-hour windows}
  \item Baseline intensity assumed locally constant
\end{itemize}

\textbf{Model used}: multi-exponential Hawkes
\end{frame}

% ------------------------------------------------------------
\begin{frame}{Estimation Strategy on Real Data}
\begin{itemize}
  \item Full MLE unstable for multi-exponential kernels
  \item Adopt a \textbf{two-step approach}:
  \begin{enumerate}
    \item Estimate decay parameters via GMM on inter-arrival times
    \item Fix decay rates and estimate remaining parameters
  \end{enumerate}
  \item Improves numerical stability
\end{itemize}
\end{frame}

% ------------------------------------------------------------
\begin{frame}{Empirical GOF Results}
\begin{itemize}
  \item Transformation-based test:
  \begin{itemize}
    \item Rejects more than naive test
    \item Provides a sensitive diagnostic
  \end{itemize}
  \item Naive RTC:
  \begin{itemize}
    \item Rejects almost systematically
    \item Extremely sensitive to plug-in estimation
  \end{itemize}
\end{itemize}
\end{frame}

% ------------------------------------------------------------
\begin{frame}{Why Does RTC Fail?}
\begin{itemize}
  \item Q–Q plots of RTC-transformed inter-arrival times
  \item On simulated data: close to Exp(1)
  \item On real data:
  \begin{itemize}
    \item Global deviations from exponentiality
    \item Not only tail effects
  \end{itemize}
\end{itemize}

\begin{block}{Interpretation}
RTC amplifies mild misspecification + estimation error
\end{block}
\end{frame}

% ------------------------------------------------------------
\begin{frame}{Conclusion}
\begin{itemize}
  \item Accounting for parameter estimation is \textbf{crucial}
  \item Transformation-based GOF:
  \begin{itemize}
    \item Correctly sized
    \item Interpretable on real data
  \end{itemize}
  \item Naive RTC can be misleading in practice
\end{itemize}

\vspace{0.2cm}
\textbf{Takeaway}: GOF tests should be used as \alert{diagnostic tools}, not mechanical decisions
\end{frame}

% ------------------------------------------------------------
\begin{frame}{Perspectives}
\begin{itemize}
  \item Data-driven choice of $(n, \tau)$
  \item Time-varying baseline intensities
  \item Multivariate Hawkes processes
  \item Estimation-aware alternatives to RTC
\end{itemize}
\end{frame}

\end{document}
